\documentclass{article}

%\usepackage[paperwidth=480pt, paperheight=1800pt]{geometry}
\usepackage[a4paper]{geometry}
%\geometry{left=3cm, right=3cm, top=2cm, bottom=4cm, footskip=1cm}

\usepackage[colorlinks=true,linkcolor=blue,urlcolor=blue,citecolor=black]{hyperref}

\usepackage{indentfirst}


\begin{document}
	\title{
		\textbf{A Hybrid Constructive Heuristic for the ROADEF/EURO Challenge 2018:\\Cutting Optimization Problem}
	}
	\author{
		Zhouxing Su, Zhipeng L\"u, Bohan Li \quad (Team ID: S28)
	}
	\date{
		SMART, School of Computer Science and Technology, Huazhong University of Science and Technology, Wuhan, China
	}
	\maketitle
	
	
	\section*{Abstract}
		
		This document presents a hybrid algorithm which integrates mixed integer fractional programming model into a greedy construction to solve the Cutting Optimization Problem posed in the ROADEF/EURO Challenge 2018\footnote{See \url{http://challenge.roadef.org/2018/en/index.php} for details.}.
	
	\section{Introduction}
		
		The topic of the ROADEF/EURO challenge is the two-dimensional guillotine cut problem with defects and sequencing constraints this year.
		The objective of this cutting optimization problem is to find a cutting pattern with minimal wasted glass in order to produce every demanded item in given size.
		Apart from the aforementioned item production constraints, defect superposing constraints and sequencing constraints, a solution to this problem must satisfy the cut depth limit and the cut gap limits.
		In this work, a hybrid constructive heuristic utilizing a mixed integer fractional programming for subproblem solving is proposed to tackle this challenging cutting optimization problem.
	
	\section{Solution Method}
	
		Our algorithm determines the cutting pattern within a single 1-cut iteratively until every item is produced.
		Partial solutions are generated at each iteration, and a complete solution is retrieved by concatenating them one by one.
		This strategy was inspired by the block building technique in packing and loading problems described in \cite{fanslau2010tree}.
		
		Specifically, a mixed integer fractional programming model is used to figure out a cutting pattern with maximal utilization ratio, i.e., the proportion of the total area covered by items to the area of the rectangle produced by the 1-cut.
		This model to be solved in each iteration take every constraints into account except that not all items are required to be produced.
		When the model is infeasible, we can infer that nothing can be put into current plate, so the algorithm will start to consider the next plate and resume the construction.
		To solve the fractional programming model in the subproblem, the parametric approach mentioned in \cite{bajalinov2013linear} is employed to transform the models into linear ones.
		
	
	\section{Computational Environment}
		
		The submitted executable needs to be run on Ubuntu 18.04.1 x64 operating system.
		It is compiled with GNU g++ 7.3 and utilizes a third party solver which is Gurobi 8.0.1 x64.
		The source code is C++14 compatible and can also be compiled with Visual Studio 2017.
		
		Our computational results were generated on Windows Server 2012 with Intel Xeon E5-2698v3 2.30GHz 32 cores and 192GB RAM.
		In our test, only 4 cores (8 threads) are used, and the timeout of each instance was set to 3600 seconds.
	
	
	\begin{thebibliography}{9}
		\bibitem{fanslau2010tree}
		Fanslau, T., \& Bortfeldt, A. (2010). A tree search algorithm for solving the container loading problem. INFORMS Journal on Computing, 22(2), 222-235.
		
		\bibitem{bajalinov2013linear}
		Bajalinov, E. B. (2013). Linear-fractional programming theory, methods, applications and software (Vol. 84). Springer Science \& Business Media.
		
	\end{thebibliography}
	
\end{document}


